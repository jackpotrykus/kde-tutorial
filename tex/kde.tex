\documentclass[11pt]{report}
\input{/Users/jackpotrykus/Documents/LaTeX/reportpreamble.tex}
\input{/Users/jackpotrykus/Documents/LaTeX/mycommands.tex}
\allowdisplaybreaks

% Header
\usepackage{lastpage}
\usemintedstyle{trac}

\begin{document}

\subsection*{What is a Kernel?}
\begin{minted}[breaklines]{r}
k <- function(x) return(dnorm(x)) # gaussian kernel
\end{minted}


\begin{minted}[breaklines]{r}
fhat <- function(x, xs, h) {
    # estimate the density at point x
    n <- length(xs)
    return(sum(k(((x - xs) / h))) / (n * h))
}
\end{minted}


\subsection*{Choosing Optimal Bandwidth}
\begin{minted}[breaklines]{r}
loocv <- function(xs, hs) {
    # calculate the leave-one-out cross-validation score for various bandwidths
    n <- length(xs)
    scores <- vector()
    for (h in hs) {
        score <- 0
        for (xi in xs) {
            for (xj in xs) {
                score <- score + dnorm((xi-xj)/h, 0, sqrt(2)) - 2 * k((xi-xj)/h)
            }
        }
        score <- score / (h * n^2)
        score <- score + 2 * k(0) / (n * h)
        scores <- c(scores, score)
    }
    return(scores)
}

optimal_h <- function(loocv_scores, hs) {
    # determine the bandwidth which corresponds to minimum loocv score
    min_cv <- min(loocv_scores)
    indexes <- which(loocv_scores %in% min_cv)
    index <- min(indexes)
    return(hs[index])
}
\end{minted}


\subsection*{Plotting the Density Estimate}
\begin{minted}[breaklines]{r}
kde <- function(xs, h) {
    # produce estimates of the density at all values of x
    vals <- vector()
    for (x in xs) {
        vals <- c(vals, fhat(x, xs, h))
    }
    return(vals)
}
\end{minted}


\end{document}
